\documentclass{article}

\usepackage[OT4]{polski}
\usepackage[utf8]{inputenc}
\usepackage{indentfirst}
\usepackage[left=2cm,right=2cm,top=1.5cm,bottom=2cm,includeheadfoot,a4paper]{geometry}
\usepackage{tabularx}
\usepackage{graphicx}
\usepackage{marginnote}
\usepackage{tabularx}
\usepackage{multicol}

\geometry{a4paper}
\linespread{1.0}
\frenchspacing
\renewcommand{\arraystretch}{1.4}

\title{
	Projektowanie i programowanie obiektowe II\\
	\Huge{Wprowadzenie do pracowni projektowania obiektowego}
}
\author{mgr inż. Krzysztof Rewak}
\date{\today}

\begin{document}
	\maketitle

	\section*{Plan zajęć}
	Przewidywany ogólny plan pierwszych zajęć laboratoryjnych: 
	\begin{itemize}
	\item przedstawienie zasad zaliczenia przedmiotu; 
	\item przedstawienie planu tematów i zajęć na cały semestr; 
	\item przedstawienie sugerowanej literatury przedmiotu.
	\end{itemize}
	
	\section*{Zasady zaliczenia przedmiotu}
	\begin{itemize}
	\item obecność na zajęciach jest obowiązkowa;
	\item dopuszczona jest jedna nieusprawiedliwiona nieobecność;
	\item każdy student zobowiązany jest do przesłania rozwiązanych zadań z list po każdych zajęciach; dotyczy to również studentów nieobecnych na danych zajęciach;
	\item wskazane listy zadań będą oceniane w skali 2-5;
	\item ocena końcowa będzie średnią wyliczoną z wszystkich ocen z semestru;
	\item aktywni na zajęciach studenci otrzymują tzw. plusy, które będą pozytywnie wpływały na końcową ocenę.
	\end{itemize}
	
	\section*{Wykaz tematów}
	\begin{itemize}
	\item Wprowadzenie do pracowni projektowania obiektowego
	\item UML: diagramy struktur
	\item UML: diagramy zachowań
	\item Techniczna dokumentacja projektu
	\item Użytkowa dokumentacja projektu
	\item Estymowanie i specyfikowanie projektów
	\item Testowanie aplikacji
	\item Podsumowanie semestru
	\end{itemize}
	
	\begin{thebibliography}{9}
	
		\bibitem{grebosz}
		Jerzy Grębosz,
		\textit{Symfonia C++ Standard},
		Wydawnictwo Edition 2000, 2008
	
		\bibitem{grebosz}
		Stanisław Wrycza, Bartosz Marcinkowski, Krzysztof Wyrzykowski
		\textit{Język UML 2.0 w modelowaniu systemów informatycznych},
		Wydawnictwo Helion, 2006
	
	\end{thebibliography}

\end{document}