\documentclass[10pt]{beamer}

\usetheme{metropolis}
\usepackage{appendixnumberbeamer}

\usepackage{booktabs}
\usepackage[scale=2]{ccicons}
\usepackage{graphicx}
\usepackage{hyperref}
\usepackage{circuitikz}
\usepackage{pdflscape}
\usepackage{smartdiagram}

\usepackage{color}
\usepackage{listings}

\lstset{
	basicstyle=\footnotesize\ttfamily,
    keepspaces=true,
    showstringspaces=false,
    language=PHP,
    commentstyle=\ttfamily,
}

\usepackage[OT4]{polski}
\usepackage[utf8]{inputenc}

\usepackage{pgfplots}
\usepgfplotslibrary{dateplot}

\usepackage{xspace}
\newcommand{\themename}{\textbf{\textsc{metropolis}}\xspace}

\setbeamertemplate{frame footer}{}
\setbeamertemplate{frame numbering}{}

\usetikzlibrary{shapes,arrows}

\tikzstyle{decision} = [diamond, draw, fill=blue!20, 
    text width=4.5em, text badly centered, node distance=3cm, inner sep=0pt]
\tikzstyle{block} = [rectangle, draw, fill=blue!20, 
    text width=5em, text centered, rounded corners, minimum height=4em]
\tikzstyle{line} = [draw, -latex']
\tikzstyle{cloud} = [draw, ellipse,fill=red!20, node distance=3cm,
    minimum height=2em]


\title{Czysty kod, część III}

\subtitle{Zaawansowane metody programowania}
\author{mgr inż. Krzysztof Rewak}
\date{\today}
\institute{Wydział Nauk Technicznych i Ekonomicznych \\ Państwowa Wyższa Szkoła Zawodowa im. Witelona w Legnicy}

\begin{document}

\maketitle

\begin{frame}{Plan prezentacji}
  \setbeamertemplate{section in toc}[sections numbered]
  \tableofcontents[hideallsubsections]
\end{frame}


\section{Brzydkie zapachy w kodzie}

\begin{frame}{Komentarze}
\begin{itemize}
	\item niewłaściwe informacje
	\item przestarzałe komentarze
	\item nadmiarowe komentarze
	\item źle napisane komentarze
	\item zakomentowany kod
\end{itemize}
\end{frame}

\begin{frame}{Środowisko}
\begin{itemize}
	\item budowanie wymaga więcej niż jednego kroku
	\item testy wymagają więcej niż jednego kroku
\end{itemize}
\end{frame}

\begin{frame}{Funkcje}
\begin{itemize}
	\item nadmiar argumentów
	\item argumenty wyjściowe
	\item argumenty znacznikowe
	\item martwe funkcje
\end{itemize}
\end{frame}

\begin{frame}{Ogólne}
\begin{itemize}
	\item wiele języków w jednym pliku źrółowym
	\item oczywiste działanie jest nieimplementowane
	\item niewłaściwe działanie w warunkach granicznych
	\item zdjęte zabezpieczenia
	\item powtórzenia
	\item kod na nieodpowiednim poziomie abstrakcji
\end{itemize}
\end{frame}

\begin{frame}{Ogólne}
\begin{itemize}
	\item klasy bazowe zależne od swoich klas pochodnych
	\item za dużo informacji
	\item martwy kod
	\item separacja pionowa
	\item niespójność
	\item zaciemnianie
\end{itemize}
\end{frame}

\begin{frame}{Ogólne}
\begin{itemize}
	\item sztuczne sprzężenia
	\item \emph{zazdrość o funkcje}
	\item argumenty wybierające
	\item zaciemnianie intecji
	\item źle rozmieszczona odpowiedzialność
	\item niewłaściwe metody statyczne
\end{itemize}
\end{frame}

\begin{frame}{Ogólne}
\begin{itemize}	
	\item użycie opisowych zmiennych
	\item nazwy funkcji powinny informowac o tym, co realizują
	\item \emph{zrozumienie algorytmu}
	\item zamiana zależności logicznych na fizyczne
	\item zastosowanie polimorfizmu zamiast if-else/switch-case
	\item wykorzystywanie standardowych konwencji
\end{itemize}
\end{frame}

\begin{frame}{Ogólne}
\begin{itemize}	
	\item wykorzystywanie standardowych konwencji
	\item zamiana magicznych liczb na nazwane stałe
	\item precyzja
	\item struktura przed konwencją
	\item hermetyzacja warunków
	\item unikanie warunków negatywnych
	\item funkcje powinny wykonywać jedną operację
\end{itemize}
\end{frame}

\begin{frame}{Ogólne}
\begin{itemize}	
	\item ukryte sprzężenia czasowe
	\item unikanie dowolnych działań
	\item hermetyzacja warunków granicznych
	\item jeden poziom abstrakcji w funkcjach
	\item dane konfigurowane na wysokim poziomie
	\item unikanie nawigacji przechodnich
\end{itemize}
\end{frame}

\begin{frame}{Nazwy}
\begin{itemize}	
	\item wybór opisowych nazw
	\item wybór nazw na odpowiednich poziomach abstrakcji
	\item standardowa nomenklatura
	\item jednoznaczne nazwy
	\item długie nazwy dla długich zakresów
	\item unikanie kodowania
	\item nazwy opisujące efekty uboczne
\end{itemize}
\end{frame}

\section{Podsumowanie}

\appendix

\begin{frame}[standout]
	Pytania?
\end{frame}

\begin{frame}{}

	Kod prezentacji dostępny jest w repozytorium git pod adresem \texttt{https://bitbucket.org/krewak/pwsz-zmp} \\ \ \\

	\begin{figure}
		\centering
		\href{https://bitbucket.org/krewak/pwsz-ppsi}{
			\includegraphics[width=.15\textwidth]{../_template/bitbucket.png}
		}
	\end{figure}
	
	Wszystkie informacje dot. kursu dostępne są pod adresem \texttt{http://pwsz.rewak.pl/kursy/10} \\ \ \\

	\begin{figure}
		\centering
		\href{http://pwsz.rewak.pl/kursy/3}{
			\includegraphics[width=.15\textwidth]{../_template/rewak.png}
		}
	\end{figure}

\end{frame}

\end{document}
