\documentclass{article}

\usepackage[OT4]{polski}
\usepackage[utf8]{inputenc}
\usepackage{indentfirst}
\usepackage[left=2cm,right=2cm,top=1.5cm,bottom=2cm,includeheadfoot,a4paper]{geometry}
\usepackage{tabularx}
\usepackage{graphicx}
\usepackage{marginnote}
\usepackage{hyperref}
\usepackage{listings}

\geometry{a4paper}
\linespread{1.0}
\frenchspacing
\renewcommand{\arraystretch}{1.4}

\title{
	Projektowanie i programowanie obiektowe \\
	\Huge{Wprowadzenie do pracowni programowania obiektowego}
}
\author{mgr inż. Krzysztof Rewak}
\date{\today}

\begin{document}
	\maketitle

	\section{Plan zajęć}
	Przewidywany ogólny plan pierwszych zajęć laboratoryjnych:
	\begin{itemize}
		\item przedstawienie zasad zaliczenia przedmiotu;
		\item przedstawienie planu tematów i zajęć na cały semestr;
		\item przedstawienie sugerowanej literatury przedmiotu;
		\item wprowadzenie do systemu kontroli wersji Git.
	\end{itemize}
	
	\section{System kontroli wersji}
	Należy utworzyć konto użytkownika na dowolnym systemie hostującym system kontroli wersji Git; zalecanym rozwiązaniem jest \href{https://bitbucket.org/product/}{Bitbucket}. Po zakończeniu procedury aktywacyjnej, należy utworzyć prywatne repozytorium o nazwie \texttt{PPO1 - Imię Nazwisko} i udostępnić je prowadzącemu dla konta RewakK@pwsz.legnica.edu.pl.
	
	\subsection{Instrukcja korzystania z systemu kontroli wersji}
	Należy sprawdzić czy Git jest zainstalowany w systemie: najlepiej uruchomić Cmder i wpisać \texttt{git --version}. Jeżeli nie jest zainstalowany, należy zainstalować, najlepiej w pakiecie Laragon, który przyda się na następnych zajęciach: \url{https://sourceforge.net/projects/laragon/files/releases/4.0/laragon-full.exe}
	
	\subsubsection{Inicjalizacja katalogu repozytorium}
	Należy utworzyć katalog repozytorium, przykładowo w następujący sposób:
	\begin{itemize}
		\item \texttt{cd /} - przechodzi do głównego katalogu, tutaj \texttt{C:}
		\item \texttt{mkdir ppo} - tworzy katalog \texttt{ppo}
		\item \texttt{cd ppo} - przechodzi do katalogu \texttt{ppo}
		\item \texttt{mkdir 123456}, gdzie \texttt{123456} to numer indeksu - tworzy katalog \texttt{123456}
		\item \texttt{cd 123456}  - przechodzi do katalogu \texttt{123456}
	\end{itemize}
	
	Polecenie \texttt{ls} pokaże istniejące w wybranej ścieżce pliki i katalogi.
	
	\subsubsection{Inicjalizacja repozytorium}
	Należy zainicjować repozytorium w następujący sposób (wewnątrz pustego katalogu projektu):
	\begin{itemize}
		\item \texttt{git init} - inicjuje repozytorium
		\item \texttt{git remote add origin https://bitbucket.org/krewak/pwsz.git} - dodaje połączenie do serwera; oczywiście należy podać adres do swojego repozytorium
		\item \texttt{git pull origin master} - ściąga wszystkie ewentualne zmiany
		\item \texttt{git status} - pokazuje stan repozytorium
	\end{itemize}
	
	\subsubsection{Wprowadzanie zmian do repozytorium}
	Wprowadzanie zmian można przeprowadzić następująco:
	\begin{itemize}
		\item \texttt{touch readme.md} - tworzy plik
		\item proszę wpisać w utworzonym pliku swój numer indeksu, email studencki, adres repozytorium oraz imię i nazwisko
		\item \texttt{git add .} - (kropka po słowie \texttt{add} oznacza, że wybieramy wszystkie pliki) dodaje do repozytorium wszystkie pliki
		\item \texttt{git status} - pokazuje stan repozytorium
		\item \texttt{git commit -am "readme"} - dodaje commit o tytule \texttt{readme}
		\item \texttt{git push origin master} - przesyła commity na serwer
	\end{itemize}

	Przed dodaniem plików do repozytorium warto dodać w katalogu głównym plik \texttt{.gitignore}, którego treścią będzie jednolinijkowiec \texttt{*.exe}. Uniemożliwi to dodawanie plików wykonawczych do repozytorium, co drastycznie zmniejszy jego wielkość.
	
	\section{Zadanie do wykonania}
	Należy do repozytorium dodać katalog \texttt{lab01} z plikiem \texttt{lab01.txt}. Plik powinien zawierać listę komend potrzebnych do zainicjowania repozytorium, pobrania zmian z serwera oraz wgrania nowych zmian.

\begin{thebibliography}{9}
	\bibitem{grebosz} 
	Jerzy Grębosz.
	\textit{Symfonia C++ Standard}. 
	Wydawnictwo Edition 2000, 2008.
	
	\bibitem{martin} 
	Robert C. Martin.
	\textit{Czysty kod. Podręcznik dobrego programisty}. 
	Wydawnictwo Helion, 2010.
\end{thebibliography}



\end{document}