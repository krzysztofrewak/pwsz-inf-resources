\documentclass[10pt]{beamer}

\usetheme{metropolis}
\usepackage{appendixnumberbeamer}

\usepackage{booktabs}
\usepackage[scale=2]{ccicons}
\usepackage{graphicx}
\usepackage{hyperref}
\usepackage{circuitikz}
\usepackage{pdflscape}
\usepackage{smartdiagram}

\usepackage{color}
\usepackage{listings}

\lstset{
	basicstyle=\footnotesize\ttfamily,
    keepspaces=true,
    showstringspaces=false,
    language=PHP,
    commentstyle=\ttfamily,
}

\usepackage[OT4]{polski}
\usepackage[utf8]{inputenc}

\usepackage{pgfplots}
\usepgfplotslibrary{dateplot}

\usepackage{xspace}
\newcommand{\themename}{\textbf{\textsc{metropolis}}\xspace}

\setbeamertemplate{frame footer}{}
\setbeamertemplate{frame numbering}{}

\usetikzlibrary{shapes,arrows}

\tikzstyle{decision} = [diamond, draw, fill=blue!20, 
    text width=4.5em, text badly centered, node distance=3cm, inner sep=0pt]
\tikzstyle{block} = [rectangle, draw, fill=blue!20, 
    text width=5em, text centered, rounded corners, minimum height=4em]
\tikzstyle{line} = [draw, -latex']
\tikzstyle{cloud} = [draw, ellipse,fill=red!20, node distance=3cm,
    minimum height=2em]


\title{Statyczne strony w internecie}

\subtitle{Zaawansowane metody programowania}
\author{mgr inż. Krzysztof Rewak}
\date{\today}
\institute{Wydział Nauk Technicznych i Ekonomicznych \\ Państwowa Wyższa Szkoła Zawodowa im. Witelona w Legnicy}

\begin{document}

\maketitle

\begin{frame}{Plan prezentacji}
  \setbeamertemplate{section in toc}[sections numbered]
  \tableofcontents[hideallsubsections]
\end{frame}


\section{HTML, szkielet każdej strony internetowej}

\begin{frame}{Czym nie jest HTML?}
	Warto zapamiętać, że przede wszystkim \textbf{HTML nie jest językiem programowania}.
\end{frame}

\begin{frame}{Czym jest HTML?}
	\textbf{HTML}, \emph{HyperText Markup Language}, to hipertekstowy język znaczników umożliwiający ustandaryzowane zapisanie struktury strony internetowej.
	
	Prostymi słowami: jest to szkielet stron internetowych.
	
	Może zostać statycznie uruchomiony z poziomu pliku *.html przez każdą przeglądarkę internetową lub wygenerowany przez system internetowy.
\end{frame}

\begin{frame}[fragile]{Prosty przykład}
	Najprotszy szkielet strony internetowej powinien wyglądać następująco:
	
	\noindent\rule{2cm}{0.4pt}

	\begin{lstlisting}
<!DOCTYPE html>
<html lang="en">

  <head>
    <meta charset="utf-8">
    <title>Homepage title</title>
  </head>

  <body>
    <h1>Title</h1>
    <p>Content</p>
  </body>

</html>
	\end{lstlisting}
\end{frame}

\begin{frame}[fragile]{Budowa tagu HTML}
	Można z grubsza podzielić tagi na dwa podstawowe rodzaje: domykane i niedomykane. Domykane zawsze muszą zostać zamknięte przez tag domknięcia:
	\begin{lstlisting}
<tag attribute1="value1" attribute2="value2"></tag>
	\end{lstlisting}
	
	Analogicznie sprawa się ma z tagami niedomykanymi:
	
	\begin{lstlisting}
<tag attribute1="value1" attribute2="value2">
	\end{lstlisting}
	
	Oprócz globalnych atrybutów (przykładowo \texttt{class}, \texttt{id} i \texttt{style}), każdy tag ma listę dostępnych dla siebie atrybutów, których każdy może przyjąć pewną wartość. Szczegóły można znaleźć w dokumentacji pod adresem \texttt{\href{https://www.w3schools.com/tags/default.asp}{https://www.w3schools.com/tags/default.asp}}.
\end{frame}

\begin{frame}[fragile]{Najpopularniejsze tagi HTML: \texttt{a}}
	Tag \texttt{<a>} pozwala na tworzenie odnośników zwanych hiperłączami. Poniższy kod doda na stronie internetowej tekst \emph{go to Google}, który po kliknięciu nań przeniesie użytkownika na stronę \texttt{https://google.com/}:

	\begin{lstlisting}
<a href="https://google.com/">go to Google</a>
	\end{lstlisting}
	
	Można wymusić, aby klikane hiperłącza były otiwerane w nowych oknach lub zakładkach przeglądarki:

	\begin{lstlisting}
<a href="https://google.com/" target="_blank">go to Google</a>
	\end{lstlisting}
\end{frame}

\begin{frame}[fragile]{Najpopularniejsze tagi HTML: \texttt{div}}
	Tag \texttt{<div>} pozwala na tworzenie kontenerów lub sekcji dzięki którym można zbudować sensowny szkielet strony internetowej. Kontener należy wyobrażać sobie bazowo jako prostokąt, który można kłaść obok innych prostokątów, pod nimi lub w nich.

	\begin{lstlisting}
<div class="row">
  <div class="column">column 1</div>
  <div class="column">column 2</div>
  <div class="column">column 3</div>
</div>
<div class="row">
  <div class="column">column 1</div>
  <div class="column">column 2</div>
</div>
	\end{lstlisting}
	
	Domyślnie kontenery mają szerokość i wysokość swojej zawartości i pojawiają się jeden pod drugim. Oczywiście można to zmienić za pomocą kaskadowych arkuszy stylu, co zostanie przeybliżone w dalszej części wykładu.
\end{frame}

\begin{frame}[fragile]{Najpopularniejsze tagi HTML: \texttt{form} i \texttt{input}}
	Tag \texttt{<form>} pozwala na dodanie formularza na stronie, natomiast \texttt{input} umożliwia tworzenie pól formularzy:

	\begin{lstlisting}
<form>
  <input type="text" name="login" placeholder="Insert login...">
  <input type="email" name="email" placeholder="Insert email...">
  <input type="password" name="password"
         placeholder="Insert password...">
  <input type="submit" value="Submit this form">
</form>
	\end{lstlisting}
	
	Powyższy kod utworzy formularz z polem tekstowym na login użytkownika, walidowanym przez przeglądarkę polem na email, zakrytym polem na hasło oraz z przyciskiem wysyłającym formularz. 
\end{frame}

\begin{frame}[fragile]{Najpopularniejsze tagi HTML: \texttt{table}}

\begin{columns}
\begin{column}{0.5\textwidth}
	Tag \texttt{<table>} pozwala na tworzenie tabeli. Każda tabela może składać się z nagłówka, ciała oraz stopki, a w nich można zamieszczać wiersze składające się z komórek. 
	
	\ \\
	
	Należy pamiętać, że tabele służą jedynie to przedstawiania danych w formie tabelarycznej. Budowanie layoutu opartego na tabelach to przestarzała technologia.
\end{column}
\begin{column}{0.5\textwidth}
\begin{lstlisting}
<table>
  <thead>
    <tr>
      <td>no.</td>
      <td>name</td>
    </tr>
  </thead>
  <tbody>
    <tr>
      <td>1</td>
      <td>Bruce</td>
    </tr>
    <tr>
      <td>2</td>
      <td>Clark</td>
    </tr>
    <tr>
      <td>3</td>
      <td>Diana</td>
    </tr>
  </tbody>
</table>
\end{lstlisting}
\end{column}
\end{columns}

\end{frame}

\begin{frame}[fragile]{Najpopularniejsze tagi HTML: \texttt{li}}

\begin{columns}
\begin{column}{0.5\textwidth}
	Tag \texttt{<li>} pozwala na tworzenie list. 
	
	\ \\
	
	Jeżeli umieścimy tagi \texttt{<li>} w tagu \texttt{<ol>} lub \texttt{<ul>}, otrzymamy kolejno listę uporządkowaną (ponumerowaną) oraz nieuporządkowaną (oznaczoną domyślnie kropkami).
\end{column}
\begin{column}{0.5\textwidth}
\begin{lstlisting}
<li>item 1</li>
<li>item 2</li>
<li>item 3</li>

<ul>
  <li>item 1</li>
  <li>item 2</li>
  <li>item 3</li>
</ul>

<ol>
  <li>item 1</li>
  <li>item 2</li>
  <li>item 3</li>
</ol>
\end{lstlisting}
\end{column}
\end{columns}

\end{frame}

\begin{frame}[fragile]{Najpopularniejsze tagi HTML: style w tekście}
	Warto zapamiętać, że pogrubianie ważnego tekstu na stronie internetowej tagiem \texttt{<b>} jest błędem. 
	
	\ \\

	\begin{lstlisting}
<strong>Important text.</strong>
<em>Emphasized text.</em>
	\end{lstlisting}
	
	\begin{lstlisting}
<b>Bold text.</b>
<i>Alternate text.</i>
	\end{lstlisting}
\end{frame}

\section{CSS, czyli jak upiększyć internet}

\begin{frame}{Czym jest CSS?}
	\textbf{CSS}, \emph{Cascading Style Sheets}, czyli kaskadowe arkusze stylu (stylów) to język znaczników wspomagający budowanie stron internetowych.
\end{frame}

\begin{frame}[fragile]{Gdzie znaleźć CSS?}
	Dodawany jest do sekcji \texttt{<body>} w jeden z dwóch sposobów: dołączenie zewnętrznego pliku (przykładowo \texttt{style.css}) lub bezpośrednie dodanie reguł w nagłówku pliku HTML:

\begin{columns}
\begin{column}{0.5\textwidth}
\begin{lstlisting}
<link rel="stylesheet"
  type="text/css"
  href="style.css">
\end{lstlisting}
\end{column}
\begin{column}{0.5\textwidth}
\begin{lstlisting}
<style type="text/css">
  html { margin: 1em; }
  body { background: Orange; }
</style>
\end{lstlisting}
\end{column}
\end{columns}

	Zalecany sposób to dołączanie zewnętrznych plików.
\end{frame}

\begin{frame}[fragile]{Selektory CSS}
	Każda reguła składa się z selektora oraz skończonego zestawu par atrybut-wartość:
	\begin{lstlisting}
selector { attribute1: value; attribute2: value; }
	\end{lstlisting}
	
	Selektor może odnosić się do trzech typów obiektów: tagów HTML, identyfikatorów obiektów oraz klas obiektów. Należy wyobrazić sobie, że szkielet strony zbudowanej w HTML można interpretować jako zbiór obiektów - DOM. 
	\begin{lstlisting}
<div id="header" class="dark-theme responsive-width">...</div>

div { padding: 5px; }
#header { width: 100%; }
.dark-theme { background: Black; color: White; }
	\end{lstlisting}
\end{frame}

\begin{frame}[fragile]{Selektory CSS}
	Selektory w CSS można dowolnie modyfikować za pomocą specjalnych operatorów \texttt{>} oraz  \texttt{.}. Przykładowo można nadać styl:
	
	\ \\
	
	$\bullet$ każdej sekcji, której bezpośredni rodzic ma klasę \texttt{container}:	
	\begin{lstlisting}
.container > section { margin-top: 3em; }
	\end{lstlisting}

	$\bullet$ każdej sekcji, której jakikolwiek rodzic ma klasę \texttt{container}:	
	\begin{lstlisting}
.container section { padding: 1em; }
	\end{lstlisting}

	$\bullet$ każdemu obiektowi który jednocześnie ma klasy \texttt{container} i \texttt{dark-theme}:
	\begin{lstlisting}
.container.dark-theme { background: Black; }
	\end{lstlisting}

	$\bullet$ tagom \texttt{header} i \texttt{footer} naraz:
	\begin{lstlisting}
header, footer { opacity: .5; }
	\end{lstlisting}
\end{frame}

\begin{frame}{Najważniejsze atrybuty i ich wartości}	
	\begin{itemize}
		\item \texttt{color: \{ Red, \#FF0000 \}} - zmienia kolor tekstu
		\item \texttt{font-family: "Arial"} - zmienia font
		\item \texttt{width: \{ 100\%, 720px, 10em \}} - zmienia szerokość elementu
		\item \texttt{margin\{-top, -right, -bottom, -left\}: \{ 10px, 1em \}} - zmienia margines zewnętrzny
		\item \texttt{padding\{-top, -right, -bottom, -left\}: \{ 10px, 1em \}} - zmienia margines wewnętrzny
		\item \texttt{display: \{ block, inline, none \}} - zmienia sposób ułożenia względem innych elementów
	\end{itemize}
\end{frame}

\begin{frame}{Gotowe i wygodne rozwiązania - frameworki CSS}
	Istnieje mnóstwo narzędzi, które ułatwiają budowanie aplikacji internetowych od strony frontendu. Warto zapoznać się z przynajmniej jednym z tych frameworków, aby ułatwić sobie pracę w przyszłości oraz poznać obecne standardy.
	
	\begin{itemize}
		\item Bootstrap - najprawdopodobniej obecnie najpopularniejszy
		\item Semantic UI
		\item Foundation
		\item Materlialize
		\item Bulma
	\end{itemize}
\end{frame}

\begin{frame}[fragile]{Przykładowy kod z użyciem frameworka CSS}
	\begin{lstlisting}
<div class="container">
  <div class="row">
    <div class="col-12 col-md-4">
      Content
    </div>
    <div class="col-12 col-md-4">
      Content
    </div>
    <div class="col-12 col-md-4">
      Content
    </div>
  </div>
</div>
	\end{lstlisting}
	
	Powyższy kod stworzy responsywną siatkę, która dla ekranów o małej rozdzielczości będzie składała się z trzech kontenerów znajdujących się pod sobą lub dla ekranów o większej rozdzielczości - trzech kontenerów ułożonych obok siebie.
\end{frame}

\section{Podsumowanie}

\begin{frame}{Bibliografia i ciekawe źródła}
  
	\begin{thebibliography}{9}
	
		\bibitem{w3}
		https://www.w3schools.com/
	
		\bibitem{bootstrap}
		https://getbootstrap.com/
	
		\bibitem{semantic}
		https://semantic-ui.com/
	
	\end{thebibliography}

\end{frame}

\appendix

\begin{frame}[standout]
	Pytania?
\end{frame}

\begin{frame}{}

	Kod prezentacji dostępny jest w repozytorium git pod adresem \texttt{https://bitbucket.org/krewak/pwsz-zmp} \\ \ \\

	\begin{figure}
		\centering
		\href{https://bitbucket.org/krewak/pwsz-ppsi}{
			\includegraphics[width=.15\textwidth]{../_template/bitbucket.png}
		}
	\end{figure}
	
	Wszystkie informacje dot. kursu dostępne są pod adresem \texttt{http://pwsz.rewak.pl/kursy/10} \\ \ \\

	\begin{figure}
		\centering
		\href{http://pwsz.rewak.pl/kursy/3}{
			\includegraphics[width=.15\textwidth]{../_template/rewak.png}
		}
	\end{figure}

\end{frame}

\end{document}
