\documentclass{article}

\usepackage[OT4]{polski}
\usepackage[utf8]{inputenc}
\usepackage{indentfirst}
\usepackage[left=2cm,right=2cm,top=1.5cm,bottom=2cm,includeheadfoot,a4paper]{geometry}
\usepackage{tabularx}
\usepackage{graphicx}
\usepackage{marginnote}
\usepackage{hyperref}
\usepackage{listings}

\geometry{a4paper}
\linespread{1.1}
\frenchspacing
\renewcommand{\arraystretch}{1.2}

\title{
	Projektowanie i programowanie obiektowe \\
	\Huge{Dziedziczenie klas}
}
\author{mgr inż. Krzysztof Rewak}
\date{\today}

\begin{document}
	\maketitle

	\section{Dziedziczenie klas w programowaniu obiektowym}	
	Należy uruchomić w dowolnym środowisku załączony program z pliku \texttt{lab05.*}. Programy we wszystkich językach wykonują dokładnie to samo zadanie: tworzą bibliotekę do której można dodawać książki.
	
	Należy przeanalizować kod, a następnie odpowiedzieć na pytania:
	\begin{itemize}
		\item jak działa dziedziczenie?
		\item czym jest modyfikator \texttt{final}?
		\item czym jest modyfikator \texttt{virtual}?
		\item czym jest modyfikator \texttt{abstract}?
		\item czym jest metoda \texttt{super()}?
		\item ile klas może dziedziczyć z jednej klasy?
		\item ile klas może być dziedziczonych w jednej klasie?
	\end{itemize}

	\section{Biblioteka}
	Należy rozszerzyć program z biblioteką:
	\begin{itemize}
		\item dodać klasy przedstawiające komiks i podręcznik;
		\item dodać możliwość dodania do biblioteki nowych typów danych;
		\item dodać możliwość wylistowania: powieści, komiksów, podręczników lub wszystkich typów.
	\end{itemize}
	
	Plik z programem należy dołączyć do repozytorium Git. Zalecane jest uporządkowanie zadań w odpowiadającym im katalogach.

\end{document}