\documentclass{article}

\usepackage[OT4]{polski}
\usepackage[utf8]{inputenc}
\usepackage{indentfirst}
\usepackage[left=2cm,right=2cm,top=1.5cm,bottom=2cm,includeheadfoot,a4paper]{geometry}
\usepackage{tabularx}
\usepackage{graphicx}
\usepackage{marginnote}
\usepackage{hyperref}
\usepackage{listings}

\geometry{a4paper}
\linespread{1.1}
\frenchspacing
\renewcommand{\arraystretch}{1.2}

\title{
	Projektowanie i programowanie obiektowe \\
	\Huge{Konstruktory}
}
\author{mgr inż. Krzysztof Rewak}
\date{\today}

\begin{document}
	\maketitle

	\section{Porównanie obiektowych języków programowania}
	Należy uruchomić w dowolny sposób i w dowolnym środowisku załączone pliki \texttt{cpp/lab03.cpp}, \texttt{php/index.php} lub \texttt{java/Main.java}. Programy wykonują dokładnie to samo zadanie: tworzą obiekt klasy \texttt{Point}, inicjalizują nim obiekt klasy \texttt{Circle}, a następnie wykorzystują jedną z metod, aby przesunąć stworzony okrąg i wypisują jego nowe koordynaty.
	
	Należy przeanalizować kod, a następnie odpowiedzieć na pytania:
	\begin{itemize}
		\item czym jest \texttt{this}?
		\item czy można zdefiniować więcej niż jeden konstruktor?
		\item kiedy i gdzie wywoływany jest destruktor obiektu?
		\item czym się różnią programy napisane w C++, Javie i PHP?
		\item czym jest operator \texttt{<<} w C++?
		\item czym jest operacja \texttt{srand(time(NULL));} w C++?
		\item co należałoby dodać do konstruktora klasy \texttt{Circle}, aby upewnić się, że obiekt będzie miał sens geometryczny?
	\end{itemize}

	\section{Konstruktory w programowaniu obiektowym}
	Należy utworzyć odpowiedni plik, którego zawartość będzie rozszerzeniem funkcjonalnym zadania:
	\begin{itemize}
		\item program umożliwi generowanie kwadratów, czyli obiektów klasy \texttt{Square};
		\item program sprawdzi, czy wygenerowany kwadrat ma sens geometryczny;
		\item program po utworzeniu i usunięciu kwadratu wypisze jego wszystkie wierzchołki w formie punktów $(x, y)$;
		\item program pozwoli na stworzenie $n$ kwadratów;
	\end{itemize}
	
	Plik z programem należy dołączyć do repozytorium Git. Zalecane jest uporządkowanie zadań w odpowiadającym im katalogach.

\end{document}