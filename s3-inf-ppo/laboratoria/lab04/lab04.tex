\documentclass{article}

\usepackage[OT4]{polski}
\usepackage[utf8]{inputenc}
\usepackage{indentfirst}
\usepackage[left=2cm,right=2cm,top=1.5cm,bottom=2cm,includeheadfoot,a4paper]{geometry}
\usepackage{tabularx}
\usepackage{graphicx}
\usepackage{marginnote}
\usepackage{hyperref}
\usepackage{listings}

\geometry{a4paper}
\linespread{1.0}
\frenchspacing
\renewcommand{\arraystretch}{1.0}

\title{
	Projektowanie i programowanie obiektowe \\
	\Huge{Hermetyzacja i enkapsulacja}
}
\author{mgr inż. Krzysztof Rewak}
\date{\today}

\begin{document}
	\maketitle

	\section{Hermetyzacja w programowaniu obiektowym}	
	Należy uruchomić w dowolnym środowisku załączony program z katalogu \texttt{s3-inf-ppo\textbackslash laboratoria\textbackslash lab04}. Programy we wszystkich językach wykonują dokładnie to samo zadanie: tworzą instancję prostej pseudogry planszowej i przeprowadzają jedną rozgrywkę.
	
	Należy przeanalizować kod, a następnie odpowiedzieć na pytania:
	\begin{itemize}
		\item jakie są zasady gry?
		\item jaki jest domyślny modyfikator dostępu do pól i metod w wybranym języku?
		\item lepiej użyć publicznego pola czy prywatnego pola z ustawionym getterem i setterem?
		\item jaki jest cel używania modyfikatora \texttt{protected}?
		\item czy enkapsujacja różni się od hermetyzacji?
		\item czym jest metoda statyczna?
		\item czym jest pętla \texttt{foreach()}?
		\item kiedy główna pętla gry się zatrzyma i jak działają wyjątki?
		\item czy lepiej jest inicjować pola klasy przy deklaracji czy w konstruktorze?
		\item czy się różni ten sam kod w różnych językach?
	\end{itemize}

	\section{Gra}
	Należy rozszerzyć program z grą tak, aby przedstawić znajomość zagadnień przedstawianych na dotychczasowych zajęciach; należą do nich idea programowania obiektowego, pola i metody, konstruktory, hermetyzacja.
	\begin{itemize}
		\item do gry przy każdym uruchomieniu zostanie dodanych $n$ graczy; $3 \leq n \leq 10$; $n$ losowe lub konfigurowalne; nazwy graczy powinny być sensownie losowane lub podawane z klawiatury;
		\item kostka powinna mieć możliwość ustwienia liczby ścian przed dodaniem do gry;
		\item gra powinna mieć możliwość ustawienia przed rozpoczęciem rozgrywki maksymalnej liczby pól;
		\item gra powinna mieć możliwość ustawienia przed rozpoczęciem rozgrywki opcjonalnej maksymalnej liczby ruchów.
	\end{itemize}

	\subsection{Podpowiedzi i wskazówki}
	\begin{itemize}
		\item należy zastanowić się które pola i metody powinny być publiczne, a które prywatne;
		\item należy zastanowić się które dane powinny być wpisywane do obiektu przez konstruktor, a jakie przez dodatkowe metody.
	\end{itemize}

	\subsection{Zadanie dla ambitnych}
	Do programu należy dopisać następujące funkcjonalności, które zwiększą dynamikę gry:
	\begin{itemize}
		\item jeżeli gracz wyrzuci kostką liczbę $k = 1$, a znajdować się będzie na pozycji nieparzystej, wówczas pionek przesunie się do tyłu o $k'$ równemu nowemu rzutowi kostką;
		\item jeżeli gracz wyrzuci kostką liczbę $k = k_{max}$, a znajdować się będzie na pozycji podzielnej przez 7, wówczas gracz otrzyma dodatkowy rzut kostką.
	\end{itemize}

\end{document}