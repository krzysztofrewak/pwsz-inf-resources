\documentclass{article}

\usepackage[OT4]{polski}
\usepackage[utf8]{inputenc}
\usepackage{indentfirst}
\usepackage[left=2cm,right=2cm,top=1.5cm,bottom=2cm,includeheadfoot,a4paper]{geometry}
\usepackage{tabularx}
\usepackage{graphicx}
\usepackage{marginnote}
\usepackage{hyperref}
\usepackage{listings}

\geometry{a4paper}
\linespread{1.1}
\frenchspacing
\renewcommand{\arraystretch}{1.2}

\title{
	Projektowanie i programowanie obiektowe \\
	\Huge{Paradygmat obiektowy \\ - modelowanie świata za pomocą obiektów}
}
\author{mgr inż. Krzysztof Rewak}
\date{\today}

\begin{document}
	\maketitle

	\section{Programowanie strukturalne a obiektowe}
	Należy skompilować i uruchomić w dowolnym środowisku programistycznym załączone pliki \texttt{lab02a.cpp} oraz \texttt{lab02b.cpp} lub uruchomić \texttt{lab02a.php} oraz \texttt{lab02b.php}. Oba programy wykonują dokładnie to samo zadanie: generują listę $n$ studentów z losowymi numerami indeksów, a następnie wypisują te numery indeksów na standardowym wyjściu. Pliki \texttt{lab02a} zostały napisane w formie strukturalnej, natomiast \texttt{lab02b} - zgodnie z regułami programowania obiektowego.
	
	Należy przeanalizować kod, a następnie odpowiedzieć na pytania:
	\begin{itemize}
		\item jaka jest różnica między klasą a obiektem?
		\item czym jest metoda?
		\item czym jest pole klasy?
		\item jakie są zalety modelowania obiektowego?
		\item jakie są różnice między C a PHP?
	\end{itemize}
	
	\section{Zadanie programistyczne}
	Należy utworzyć plik \texttt{lab02}, którego zawartość będzie rozszerzeniem funkcjonalnym załączonych do zadania programów lub w dowolnym języku obiektowym stworzyć nowy program, w którym:
	\begin{itemize}
		\item student - poza numerem indeksu - powinien posiadać też imię i nazwisko; proponowanym rozwiązaniem jest utworzenie tablice stringów, które będą przechowywały przykładowe imiona i nazwiska oraz metody, która będzie je losowała i przydzielała studentom;
		\item student - poza numerem indeksu, imieniem i nazwiskiem - powienien móc mieć ustawiony status studenta; proponowanym rozwiązaniem jest utworzenie binarnej zmiennej, która będzie wskazywała czy student jest aktywny, czy nie; można wykorzystać metodę \texttt{rand()}, aby losowo dopasować każdemu studentowi jego status;
		\item program powinien na końcu wypisać na standardowym wyjściu listę wygenerowanych aktywnych studentów w formie \texttt{nazwisko imię (numer indeksu)}.
	\end{itemize}
	
	Plik z programem należy dołączyć do repozytorium Git. Zalecane jest uporządkowanie zadań w odpowiadającym im katalogach. 

\end{document}