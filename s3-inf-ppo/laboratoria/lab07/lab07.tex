\documentclass{article}

\usepackage[OT4]{polski}
\usepackage[utf8]{inputenc}
\usepackage{indentfirst}
\usepackage[left=2cm,right=2cm,top=1.5cm,bottom=2cm,includeheadfoot,a4paper]{geometry}
\usepackage{tabularx}
\usepackage{graphicx}
\usepackage{marginnote}
\usepackage{hyperref}
\usepackage{listings}

\geometry{a4paper}
\linespread{1.1}
\frenchspacing
\renewcommand{\arraystretch}{1.2}

\title{
	Projektowanie i programowanie obiektowe \\
	\Huge{Interfejsy w programowaniu obiektowym}
}
\author{mgr inż. Krzysztof Rewak}
\date{\today}

\begin{document}
	\maketitle

	\section{Interfejsy w programowaniu obiektowym}
	Należy uruchomić w dowolnym środowisku załączony program z katalogu \texttt{lab07}. Programy we wszystkich językach wykonują dokładnie to samo zadanie.
	
	\subsection{Java}
	W katalogu \texttt{lab07/java} znajduje się kod źródłowy aplikacji podzielonej na pakiety. Należy zbudować, skompilować i uruchomić projekt w dowolny sposób. Zalecane jest środowisko IntelliJ IDEA (dostępne dla studentów za darmo).
	
	\subsection{PHP}
	W katalogu \texttt{lab07/php} znajduje się kod źródłowy aplikacji wymagającej autoloadera. Należy uruchomić komendę \texttt{composer install}, aby utworzyć katalog \texttt{vendor} z wymaganymi plikami. Aplikację uruchamia się poprzez polecenie \texttt{php index.php} w głównym katalogu. Zalecane jest środowisko PHPStorm (dostępne dla studentów za darmo).
	
	\subsection{Pytania do zadania}
	Należy odpowiedzieć na następujące pytania dotyczące interfejsów:
	\begin{itemize}
		\item czym jest interfejs?
		\item jakie metody musi implementować każda klasa, której obiekt będzie podjeżdżał pod bramę parkingu?
		\item ile interfejsów może być zaimplementowych w jednej klasie?
		\item czy interfejsy są dziedziczone?
		\item co się stanie jeżeli pod bramę podjedzie czołg klasy \texttt{Tank}?
		\item czy strażnik wpuści dwa samochody o takim samym numerze rejestracyjnym? A powinien?
	\end{itemize}
	
	A także dotyczące technikaliów:
	\begin{itemize}
		\item jak działają przestrzenie nazw w PHP i pakiety w Javie?
		\item czy podział programu w konwencji \emph{jedna klasa - jeden plik} jest czytelny?
		\item dlaczego w pliku \texttt{.gitignore} dodano foldery \texttt{.idea} oraz \texttt{vendor} dla PHP i \texttt{out} dla Javy?
	\end{itemize}

	\section{Zadanie programistyczne}
	Należy rozszerzyć program tak, aby przedstawić znajomość zagadnień przedstawianych na dotychczasowych zajęciach (dziedziczenie oraz interfejsy), a także z poprzednich (pola i metody, konstruktory, enkapsulacja, hermetyzacja).
	
	\begin{itemize}
		\item generator kolejki powinien być konfigurowalny, tak, aby generować konkretne zestawy danych;
		\item generator powinien uruchomić się kilkukrotnie \emph{w ciągu dnia}; najlepiej zrealizować to iterowaniem po pętli z godzinami wejścia;
		\item każde wejście powinno zostać odnotowane razem z czasem wejścia;
		
		\item parking ma określony limit miejsc parkingowych, zatem parkingowy nie powinien wpuścić więcej samochodów niż ma miejsc;
		\item parking powinien być płatny, analogicznie do parkingu przy kampusie; pracownicy uczelni wjeżdżają bez opłat (chciałbym!), natomiast prawie wszyscy inni płacą określoną cennikiem sumę;
		\item na kampus mogą wjechać bez opłat samochody dostawcze kurierów oraz pojazdy uprzywilejowane, ale nie wliczają się do liczby zajętych miejsc parkingowych;
		\item na kampus mogą wjechać bez opłat rowerzyści, ale jedynie w liczbie nieprzekraczającej określonej liczby stojaków na rowery;
		\item na kampus nie mogą wjechać czołgi, ponieważ są zbyt szerokie;
		
		\item na koniec dnia wyświetlana jest uzbierana przez parkingowego kwota;
		\item parkingowy posiada \emph{czarną listę} tablic rejestracyjnych, których właścicieli nie wpuszcza na teren parkingu;
	\end{itemize}
	
	Plik z programem należy dołączyć do repozytorium Git. Zalecane jest uporządkowanie zadań w odpowiadającym im katalogach. Prezesłany program zostanie oceniony, a ocena będzie częścią składową oceny semestralnej.
	
	\subsection{Zadanie dla ambitnych}
	Należy zastanowić się jak sensownie zaimplementować mechanizm wypuszczania obiektów z parkingu. Idealnym wyjściem byłoby, gdyby o godzinie 23:00 kampus był pusty, a więc wypadałoby zmusić wszyskich - za wyjątkiem pracowniczych samochodów z wykupionym abonamentem na nocne parkowanie - do opuszczenia terenu. 

\end{document}