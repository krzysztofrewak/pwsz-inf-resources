\documentclass{article}

\usepackage[OT4]{polski}
\usepackage[utf8]{inputenc}
\usepackage{indentfirst}
\usepackage[left=2cm,right=2cm,top=1.5cm,bottom=2cm,includeheadfoot,a4paper]{geometry}
\usepackage{tabularx}
\usepackage{graphicx}
\usepackage{marginnote}
\usepackage{tabularx}
\usepackage{multicol}
\usepackage{hyperref}

\geometry{a4paper}
\linespread{1.0}
\frenchspacing
\renewcommand{\arraystretch}{1.4}

\title{
	Projektowanie i programowanie obiektowe II\\
	\Huge{Testowanie aplikacji}
}
\author{mgr inż. Krzysztof Rewak}
\date{\today}

\begin{document}
	\maketitle

	\section{TDD i testowanie aplikacji}
	Należy odpowiedzieć na pytania:
	\begin{itemize}
		\item czym jest test jednostkowy?
		\item czym jest asercja?
		\item czym jest pokrycie testami?
		\item kiedy powinny być uruchamiane testy?
		\item jaki jest sens automatyzacji testowania oprogramowania?
		\item jaki jest sens pracy w metodyce TDD?
	\end{itemize}
	
	\section{Zadanie}
	Należy:
	\begin{itemize}
		\item pobrać program ppo/lab11: \url{https://bitbucket.org/krewak/pwsz-ppo2/src/master/lab11/},
		\item pobrać wymagane zależności i uruchomić aplikację w przeglądarce,
		\item uruchomić testy jednostkowe komendą \texttt{.\textbackslash vendor\textbackslash bin\textbackslash phpunit} i sprawdzić ich status
	\end{itemize}
	
	Następnie:
	\begin{itemize}
		\item utworzyć sensowne testy do wszystkich innych klas i metod aplikacji,
		\item gotowy projekt wraz z testami dołączyć do repozytorium Git.
	\end{itemize}
	
	Ewentualnie dodać mechanizm testowania jednostkowego i stworzyć sensowne testy dla wybranego z programów (od laboratorium piątego i wzwyż) zaimplementowanych na zajęciach z Projektowania i programowania obiektowego I.
	
	\begin{thebibliography}{9}
	
		\bibitem{assert}
		\url{https://phpunit.readthedocs.io/en/7.1/assertions.html}
	
	\end{thebibliography}

\end{document}